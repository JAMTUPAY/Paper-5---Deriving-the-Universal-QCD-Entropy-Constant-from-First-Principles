\documentclass[11pt]{article}

\usepackage[margin=1in]{geometry}
\usepackage{microtype}
\usepackage{amsmath,amssymb,amsfonts}
\usepackage{booktabs}
\usepackage{enumitem}
\usepackage{xcolor}
\usepackage{hyperref}
\usepackage{siunitx}
\usepackage{graphicx}
\usepackage{mathtools}
\usepackage{bm}
\usepackage{cite}

\hypersetup{
  colorlinks=true,
  linkcolor=black,
  citecolor=black,
  urlcolor=blue
}

\title{\vspace{-0.5em}\textbf{Deriving the Universal QCD Entropy Constant from First Principles}\\[0.25em]
\large Closing the Loop from Anomalies to Observables}
\author{Johann Anton Michael Tupay\\
\small London, United Kingdom\\
\small \texttt{jamtupay@icloud.com}\\
\small ORCID: \href{https://orcid.org/0009-0008-7661-8698}{0009-0008-7661-8698}}
\date{\small\today}

\begin{document}
\maketitle

\begin{abstract}
\noindent
We derive the universal QCD RG--entropy drop used across Papers~1--4 directly from continuum field theory. Using the sphere--to--hyperbolic (Casini--Huerta--Myers, CHM) mapping and the $4$D $A$-type trace anomaly, we show that the RG--integrated entanglement loss obeys
\[
\big|\Delta S_{\rm RG}\big| \;=\; \kappa\,[a_{\rm UV}-a_{\rm IR}]\,k_B,\qquad \kappa=2\pi.
\]
For confining $SU(3)$ with a gapped IR ($a_{\rm IR}=0$) and two effectively massless Dirac flavors on the spherical trajectory ($N_f^{\rm eff}=2$), free-field anomaly counting gives
\[
a_{\rm UV}=(N_c^2-1)\,\frac{31}{180} + (N_c N_f^{\rm eff})\,\frac{11}{360}=\frac{281}{180},
\]
hence
\[
\big|\Delta S_{\rm RG}\big|=\frac{281\pi}{90}\,k_B=9.809\,k_B\;\approx\;9.81\,k_B.
\]
Two independent checks---an RG gradient/sum-rule identity and a thermal calculation on $\mathbb{H}^3{\times}S^1$---reproduce the same constant. The result matches the empirical value extracted in Paper~1 and used, without tuning, in Papers~2--4 for exotica, QGP onset, and neutron-star observables.
\end{abstract}

\section{Scope and main claim}
We consider asymptotically free, confining QCD with $N_c=3$ and an IR mass gap. Define the RG--integrated entanglement drop
\begin{equation}
\big|\Delta S_{\rm RG}\big|\;\equiv\;\int_{\mu_{\rm IR}}^{\mu_{\rm UV}}\frac{\partial S_{\rm EE}(\mu)}{\partial\ln\mu}\,d\ln\mu,
\label{eq:def}
\end{equation}
then using the CHM map and the $A$-type anomaly we prove
\begin{equation}
\big|\Delta S_{\rm RG}\big| = 2\pi\,[a_{\rm UV}-a_{\rm IR}]\,k_B,\qquad a_{\rm IR}=0.
\label{eq:main}
\end{equation}
Evaluating $a_{\rm UV}$ for free gluons and $N_f^{\rm eff}$ effectively massless Dirac quarks,
\begin{equation}
a_{\rm UV}=(N_c^2{-}1)\frac{31}{180}+(N_c N_f^{\rm eff})\frac{11}{360},
\end{equation}
gives for $(N_c,N_f^{\rm eff})=(3,2)$:
\[
\big|\Delta S_{\rm RG}\big|=\frac{281\pi}{90}\,k_B=9.809\,k_B.
\]
This equals the constant inferred in Paper~1 and used unchanged in Papers~2--4.

\paragraph{Assumptions and normalizations.}
(i) \textbf{CHM mapping}: vacuum on a ball $\mathcal{B}_R$ is thermal on $\mathbb{H}^3\times S^1$ at $T_0=1/(2\pi R)$; universal pieces map faithfully~\cite{CHM,Solodukhin,CHfree}. (ii) \textbf{Gapped IR}: confinement implies $a_{\rm IR}=0$. (iii) \textbf{UV content}: on the spherical trajectory, $u,d$ count as massless while $s$ is treated heavy ($N_f^{\rm eff}{=}2$); taking $N_f^{\rm eff}{=}3$ would give $10.385\,k_B$. (iv) \textbf{Renormalized hyperbolic volume}: normalized to unity.

\section{Derivation I: entanglement--anomaly route}
The CHM map relates the sphere entanglement to thermal physics on $\mathbb{H}^3\times S^1$; the $A$-type anomaly controls the $\mu$-derivative of $-\beta F$ under smooth deformations. One finds~\cite{CHM,Solodukhin,MyersSinha}
\begin{equation}
\frac{\partial S_{\rm EE}}{\partial\ln\mu}=2\pi\,a(\mu)\,,
\end{equation}
so integrating Eq.~\eqref{eq:def} yields Eq.~\eqref{eq:main}. Using free-field anomaly coefficients~\cite{Duff,Anselmi}
\begin{equation}
a_{\rm vector}=\frac{31}{180},\quad a_{\rm Dirac}=\frac{11}{360},
\end{equation}
and multiplicities $N_{\rm vec}=N_c^2{-}1$, $N_{\rm Dirac}=N_c N_f^{\rm eff}$ gives the stated $9.809\,k_B$ for $(3,2)$.

\section{Derivation II: RG gradient / sum-rule cross-check}
The $4$D $a$-theorem implies $da/d\ln\mu\le0$~\cite{KomargodskiSchwimmer,Cardy4D,JackOsborn,FreedmanOsborn}. Differentiating $\partial S_{\rm EE}/\partial\ln\mu=2\pi a(\mu)$ and integrating by parts,
\begin{equation}
\big|\Delta S_{\rm RG}\big|=2\pi\,[a_{\rm UV}-a_{\rm IR}]\,k_B,
\end{equation}
independent of the detailed dynamics along the flow.

\section{Derivation III: thermal mapping cross-check}
On $\mathbb{H}^3\times S^1$ at $T_0=1/(2\pi R)$, the universal entanglement equals the thermal entropy at $T_0$; the anomaly governs the $\mu$-dependence of $F$~\cite{CHM,Solodukhin,CHfree}. Evaluating the UV and IR endpoints immediately reproduces Eq.~\eqref{eq:main} and hence $9.809\,k_B$ for $(N_c,N_f^{\rm eff})=(3,2)$.

\section{Consistency with Papers 1--4}
The derived constant equals the empirical value $9.81\pm0.29\,k_B$ extracted from a lattice-inspired $c$-function integral in Paper~1 and underpins Papers~2--4 (exotic hadrons, QGP threshold, neutron-star bounds) without tuning~\cite{TupayP1,TupayP2,TupayP3,TupayP4}.

\section{Numerical summary and flavor dependence}
\begin{center}
\begin{tabular}{@{}lcc@{}}
\toprule
Effective flavors & $a_{\rm UV}$ & $\big|\Delta S_{\rm RG}\big|$ \\
\midrule
$N_f^{\rm eff}=2$ & $281/180$ & $\dfrac{281\pi}{90}\,k_B\;=\;9.809\,k_B$ \\
$N_f^{\rm eff}=3$ & $119/72$  & $\dfrac{119\pi}{36}\,k_B\;=\;10.385\,k_B$ \\
\bottomrule
\end{tabular}
\end{center}
Per added massless Dirac flavor at fixed $N_c{=}3$,
\[
\delta\big|\Delta S_{\rm RG}\big|
= 2\pi\,(N_c a_{\text{Dirac}})\,k_B
= \frac{11\pi}{60}\,k_B \approx 0.576\,k_B.
\]

\section*{Conclusions}
A geometric identity plus free-field anomaly counting fixes
\[
\big|\Delta S_{\rm RG}\big|=9.809\,k_B\;\approx\;9.81\,k_B
\]
for QCD in the physically relevant setup. This explains the scheme independence of the constant, predicts its flavor dependence, and provides a parameter-free bridge from anomalies to the observables modeled in Papers~1--4.

\appendix

\section*{Appendix A: sphere--hyperbolic map and $\kappa=2\pi$}
For a ball $\mathcal{B}_R\subset\mathbb{R}^{3,1}$ the reduced density matrix is thermal on $\mathbb{H}^3\times S^1$ at $T_0=1/(2\pi R)$ with modular Hamiltonian
\[
K=2\pi \int_{\mathcal{B}_R}\! d^3x\,\frac{R^2-r^2}{2R}\,T_{00}(x).
\]
Under smooth deformations of $\mathbb{H}^3\times S^1$, the renormalized free energy obeys
\[
\frac{\partial}{\partial\ln\mu}\!\left[-\beta F_{\mathbb{H}^3}(\beta)\right]=2\pi\,\mathcal{V}^{\rm ren}_{\mathbb{H}^3}\,a(\mu),
\]
and with $\mathcal{V}^{\rm ren}_{\mathbb{H}^3}=1$ one finds $\partial S_{\rm EE}/\partial\ln\mu=2\pi a(\mu)$ and hence $\kappa=2\pi$~\cite{CHM,Solodukhin}.

\section*{Appendix B: free-field anomaly coefficients and multiplicities}
The standard $4$D $a$-anomaly coefficients are~\cite{Duff,Anselmi}
\[
a_{\text{real scalar}}=\tfrac{1}{360},\;
a_{\text{Weyl}}=\tfrac{11}{720},\;
a_{\text{Dirac}}=\tfrac{11}{360},\;
a_{\text{vector}}=\tfrac{31}{180}.
\]
For $SU(N_c)$ with $N_f^{\rm eff}$ Dirac fundamentals:
\[
N_{\rm vec}=N_c^2-1,\qquad N_{\rm Dirac}=N_c N_f^{\rm eff},
\]
so
\[
a_{\rm UV}=(N_c^2{-}1)\frac{31}{180}+(N_c N_f^{\rm eff})\frac{11}{360},\qquad
\big|\Delta S_{\rm RG}\big|=2\pi\,a_{\rm UV}\,k_B.
\]

\section*{Acknowledgments}
I thank colleagues for feedback on anomaly normalizations and RG monotonicity. Any remaining errors are mine.

% -------- References --------
\begin{thebibliography}{99}\setlength{\itemsep}{0.25em}

% Series papers
\bibitem{TupayP1}
J.~A.~M. Tupay, \emph{Universal Entropy--Mass Relation in QCD: Discovery from Lattice c-Function} (2025).
Zenodo \href{https://doi.org/10.5281/zenodo.16743904}{10.5281/zenodo.16743904}.

\bibitem{TupayP2}
J.~A.~M. Tupay, \emph{Entropy-Forbidden Exotic Hadrons: Universal Constraints from QCD Information Flow} (2025).
Zenodo \href{https://doi.org/10.5281/zenodo.16752674}{10.5281/zenodo.16752674}.

\bibitem{TupayP3}
J.~A.~M. Tupay, \emph{Universal Entropy Threshold for Quark--Gluon Plasma Formation} (2025).
Zenodo \href{https://doi.org/10.5281/zenodo.16762323}{10.5281/zenodo.16762323}.

\bibitem{TupayP4}
J.~A.~M. Tupay, \emph{Entropy-Constrained Neutron Stars from a Universal QCD Bound} (2025).
Zenodo \href{https://doi.org/10.5281/zenodo.16783040}{10.5281/zenodo.16783040}.

% Core field-theory / anomaly / CHM mapping
\bibitem{CHM}
H.~Casini, M.~Huerta, and R.~C.~Myers, \emph{Towards a derivation of holographic entanglement entropy}, JHEP \textbf{1105} (2011) 036.

\bibitem{KomargodskiSchwimmer}
Z.~Komargodski and A.~Schwimmer, \emph{On renormalization group flows in four dimensions}, JHEP \textbf{1112} (2011) 099.

\bibitem{Cardy4D}
J.~L.~Cardy, \emph{Is there a c-theorem in four dimensions?}, Phys.\ Lett.\ B \textbf{215}, 749 (1988).

\bibitem{Zamolodchikov}
A.~B.~Zamolodchikov, \emph{Irreversibility of the flux of the renormalization group in a 2D field theory}, JETP Lett.\ \textbf{43}, 730 (1986).

\bibitem{Solodukhin}
S.~N.~Solodukhin, \emph{Entanglement entropy, conformal anomalies and RG}, Phys.\ Lett.\ B \textbf{665}, 305 (2008).

\bibitem{Duff}
M.~J.~Duff, \emph{Twenty years of the Weyl anomaly}, Class.\ Quantum Grav.\ \textbf{11}, 1387 (1994).

\bibitem{Anselmi}
D.~Anselmi, D.~Z.~Freedman, M.~T.~Grisaru, A.~A.~Johansen, \emph{Nonperturbative formulas for central functions}, Nucl.\ Phys.\ B \textbf{526}, 543 (1998).

\bibitem{FreedmanOsborn}
D.~Z.~Freedman and H.~Osborn, \emph{Constructing a c-function for field theories}, Phys.\ Lett.\ B \textbf{432}, 353 (1998).

\bibitem{JackOsborn}
I.~Jack and H.~Osborn, \emph{Analogs for the c-theorem in four-dimensional field theories}, Nucl.\ Phys.\ B \textbf{343}, 647 (1990).

\bibitem{CHfree}
H.~Casini and M.~Huerta, \emph{Entanglement entropy in free quantum field theory}, J.\ Phys.\ A \textbf{42}, 504007 (2009).

\bibitem{MyersSinha}
R.~C.~Myers and A.~Sinha, \emph{Holographic c-theorems in arbitrary dimensions}, JHEP \textbf{1101}, 125 (2011).

\bibitem{LiuMezei}
H.~Liu and M.~Mezei, \emph{A refinement of entanglement entropy and the number of degrees of freedom}, JHEP \textbf{1304}, 162 (2013).

\bibitem{RyuTakayanagi}
S.~Ryu and T.~Takayanagi, \emph{Holographic derivation of entanglement entropy}, Phys.\ Rev.\ Lett.\ \textbf{96}, 181602 (2006).

% Optional context / reviews
\bibitem{OsbornPetkou}
H.~Osborn and A.~C.~Petkou, \emph{Implications of conformal invariance in field theories}, Annals Phys.\ \textbf{231}, 311 (1994).

\bibitem{CasiniTesteTorroba}
H.~Casini, E.~Teste, and G.~Torroba, \emph{Markov property of the vacuum and the renormalization group}, JHEP \textbf{1703}, 089 (2017).

\bibitem{Swingle}
B.~Swingle, \emph{Entanglement renormalization and holography}, Phys.\ Rev.\ D \textbf{86}, 065007 (2012).

\bibitem{Srednicki}
M.~Srednicki, \emph{Entropy and area}, Phys.\ Rev.\ Lett.\ \textbf{71}, 666 (1993).

\end{thebibliography}

\end{document}
